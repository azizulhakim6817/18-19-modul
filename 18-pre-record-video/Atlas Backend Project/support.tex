####<<< Class-22 >>>> userOTP >>>>>>>>>>>>>>>>>>>>>>>>>>>>>>>>>>>>>>>>>>>>>>>>>>>>>>>>>>>>>>>>>>>>>>>>>>>><<<<<<
1. .count("total) => console.log(total)  => count is not a function

2. controllers.js file ...............................

exports.VerifyLogin = async (req, res) => {
    let result = await VerifyOTPSevice(req);
    return res.status(200).json(result);
  };

---------------------- 1nd problems => erro ase? -----------------------------

const VerifyOTPSevice = async (req) => {
  let email = req.params.email;
  console.log(email);

  let otp = req.params.otp;
  console.log(otp);

  let total = await UserModel.find({ email: email, otp: otp }).count("total");
  console.log(total);

  return { status: "success", message: total };
};


2.   let total = await UserModel.countDocuments({ email: email, otp: otp });
     => result postman ............
     {
    "status": "success",
    "message": 1
     }


---------------------- 2nd problems => erro ase? -----------------------------

const VerifyOTPSevice = async (req) => {
  try {
    let email = req.params.email;
    let otp = req.params.otp;

    // User Count
    let total = await UserModel.find({ email: email, otp: otp }).count("total");
    if (total.lenght === 1) {
      // User ID Read
      let user_id = await UserModel.find({ email: email, otp: otp });

      // User Token Create
      let token = EncodeToken(email, user_id[0]["_id"].toString());

      // OTP Code Update To 0
      await UserModel.updateOne({ email: email }, { set: { otp: "0" } }); //set doller symble must be...

      return { status: "success", message: "Valid OTP", token: token };
    } else {
      return { status: "fail", message: "Invalid OTP" };
    }
  } catch (e) {
    return { status: "fail", message: "Invalid OTP" };
  }
};

----------- 3nd problems =>  -----------------------------

1....6060* ken?......................................
expires: new Date(Date.now() + 24 * 6060 * 1000),

2..........res.cookie ==> ......................................

exports.VerifyLogin = async (req, res) => {
  let result = await VerifyOTPSevice(req);
  if (result["status"] === "success") {
    // Cookies Option // postman(req) =>  "status": "success",
    let cookieOption = {
      expires: new Date(Date.now() + 24 * 6060 * 1000),
      httpOnly: false,
    };
    // Set Cookies With Response
    res.cookie("token", result["token"], cookieOption); 

    return res.status(200).json(result);
  } else {
    return res.status(200).json(result);
  }
};

####<<< Class-23 >>>> userOTP >>>>>>>>>>>>>>>>>>>>>>>>>>>>>>>>>>>>>>>>>>>>>>>>>>>>>>>>>>>>>>>>>>>>>>>>>>>><<<<<<
3. postman request ............................hyna...........
//UserLogout............................................................
exports.UserLogout = async (req, res) => {
  let cookieOption = {
    expires: new Date(Date.now() - 24 * 6060 * 1000),
    httpOnly: false,
  };
  res.cookie("token", "", cookieOption); // Set Cookies With Response
  return res.status(200).json({ status: "success" });
};

####<<< Class-24 >>>> userOTP >>>>>>>>>>>>>>>>>>>>>>>>>>>>>>>>>>>>>>>>>>>>>>>>>>>>>>>>>>>>>>>>>>>>>>>>>>>><<

1. undefined ase? ///////////////////////////////////////
body-parser deprecated undefined extended: provide extended option app.js:37:17
Server is running at http://localhost:8000
db is connected

=> >>> code app.js ...............
app.use(express.json({ limit: "50mb" }));
app.use(express.urlencoded({ limit: "50mb" }));

1 What is set:true?  /////////////////////////////////////////////////////

//Update Service list .............................................
const UpdateCartListService = async (req) => {
  try {
    let user_id = req.headers.user_id;
    let cartID = req.params.cartID;
    let reqBody = req.body;
    await CartModel.updateOne(
      { _id: cartID, userID: user_id },
      { set: reqBody }
    );
    return { status: "success", message: "Cart List Update Success" };
  } catch (e) {
    return { status: "fail", message: "Something Went Wrong !" };
  }
};


####<<< Class-28 >>>> userOTP >>>>>>>>>>>>>>>>>>>>>>>>>>>>>>>>>>>>>>>>>>>>>>>>>>>>>>>>>>>>>>>>>>>>>>>>>>
//..............Step-: 7 prepare SSL Payment..................... .................................
1. ...................what?.....................
let from = new FormData();
form.append("store_id", PaymentSittingModel[0]["store_id"]);
form.append("total_amount	", payable.toString());

